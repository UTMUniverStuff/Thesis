\newcommand{\texta}{Helpful\\[-1ex] \tiny{(to achieve the objective)}}
\newcommand{\textb}{Harmful\\[-1ex] \tiny{(to achieve the objective)}}
\newcommand{\textcn}{Internal origin\\[-1ex] \tiny{(product\slash company attributes)}}
\newcommand{\textdn}{External origin\\[-1ex] \tiny{(environment\slash market attributes)}}


\begin{table}
   \caption{SWOT Analysis}
   
   \begin{tikzpicture}[
       any/.style={draw,minimum width=8cm,minimum height=8cm,%
                    text width=7.5cm,align=center,outer sep=0pt},
       header/.style={any,minimum height=1cm,fill=black!10},
       leftcol/.style={header,rotate=90}
   ]
   
   
   \matrix (SWOT) [matrix of nodes,nodes={any,anchor=center},%
                   column sep=-\pgflinewidth,%
                   row sep=-\pgflinewidth,%
                   row 1/.style={nodes=header},%
                   column 1/.style={nodes=leftcol},
                   inner sep=0pt]
   {
                & {\texta} & {\textb} \\
    {\textcn} & \scriptsize \textbf{Strengths} 
    
    \begin{enumerate}
    \item [--] Ability to access multiple camera controlling systems from a browser at the same time
    \item[--] Ability to mend gaps within the current public transportation system through a supplementary mode of travel between transit lines and stops;
    
    \item[--] Stimulates environmental awareness among the community by providing a sustainable means of transport and promotes multi- and intermodal travel among residents. Multimodal travel involves the use of different modes for different trips while intermodal refers to the use of various modes in one trip
    chain;
    
    \item[--] Provides a cost-effective measure for the City to improve infrastructure with limited impact to the natural environment;
    
    \item[--] Health benefits associated with clean air will occur overtime as air quality is improved.
    
    \end{enumerate}
    
    & \scriptsize \textbf{Weakness} 
    
    \begin{enumerate}
    \item[--] Implementation may take years from initial proposal since San Fernando Valley compromises of several municipalities and approval from various departments governing each city will be required;
    
    \item[--] Funding issues exists across different jurisdictions and municipal borders even if a private vendor is financing and operating the bike-share program without assistance from city funds;
    
    \item[--] Steep, undulating streets in some areas of the Valley can physically inhibit bicyclists and deter them from bike-traveling;
    
    \item[--] Wide streets encourage drivers to speed beyond the limit and pose danger to bicyclists even with designated bike lanes in place.
    
    \end{enumerate}
    
    \\
    {\textdn} & \scriptsize \textbf{Opportunities} 
    
    \begin{enumerate}
    \item[--] Use of modular forms and sustainable technology such as solar power to allow bike stations to stand alone (without requiring power from The Grid);
    
    \item[--] Station can be disassembled and relocated to another if one location exhibits low ridership;
    
    \item[--] May increase exposure for retail storefronts in the Valley since people are passing by at slower speeds;
    
    \item[--] Promotes physical activity within the community, leading to improved health of overall community.
    
    \end{enumerate}
    
    & \scriptsize \textbf{Threats} 
    
    \begin{enumerate}
    \item[--] Lack of profitability for the selected private vendor who supplies all the components that goes into a bike-sharing system;
    
    \item[--] May be difficult to shift people’s commuting patterns;
    
    \item[--] Other bike-share companies seeking to introduce their programs to the City may threaten the selected vendor;
    
    \item[--] Cities within Chisinau may end up using different suppliers, making it difficult to standardize the policy in the region.
    
    \end{enumerate}
    
    \\
   };
   \end{tikzpicture}
   \label{swot}
\end{table}